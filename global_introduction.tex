\chapter{Introduction} \label{chap:global_intro}
The visual analysis of bodily fluids and tissues focused on diagnosing diseases using a microscope is called microscopic pathology, that is a sub-discipline of pathology. This kind of analysis still constitutes the final step to confirm if and which illness is present. Traditionally, \textit{cytopathology} and \textit{histopathology} compose microscopical pathology. Cytopathology refers to diagnosis based on the study of cytological images, that are characterized by the presence of single cells and cell clusters, while histopathology relates to diagnosis based on the study of histological images and involves examination of entire human tissues composed of an association of cells into structures which deal with a particular function.

The measurements and characterization of cells from cytological images can be performed automatically since the late 1950s, when Coulter \cite{Coulter} developed a method for sizing and counting cells, using electrical impedance directly from the blood sample. Nowadays the technique proposed by Coulter has been improved to analyze different particles. A further improvement of this approach is the Flow Cytometry,  used to measure and examine multiple physical characteristics, chemical properties simultaneously and defines the maturation stage of particles, as they flow in a fluid stream rapidly and they pass one-at-a-time through at least one laser. Particle components are fluorescently labeled and then excited by the laser to emit light at varying wavelengths, and then distinguished using an optical-to-electronic coupling system that records the way in which the cell emits fluorescence and scatters incident light from the laser. The properties measured include size, morphology, granularity and internal and external structure of cells in question. This system, due to its complexity, needs many quality controls. Some of these controls are performed internally by the same instrument, but others must be executed externally to check the performance of each component.

\section{CAD - Computer Aided Diagnosis}
The microscope is still an essential tool to the pathology laboratory today since pathologists continue to perform manual observation of samples. It can be used to check the results from an instrument or if a recalibration is needed. The manual microscopic examination involves numerous drawbacks, in particular, the results accuracy heavily depends on the operator skills. The operators develop their skills during a complex training period analyzing as many cases as possible of different pathology. Nevertheless many cases require different experts and technical opinions to reduce human error. The process of manual microscopic observation is prolonged and time-consuming, in particular, if it involves different operators for a single diagnosis. Digital microscopes are becoming routine pieces of equipment in laboratories, being a combination of a digital camera and a lens, can scan the samples and store the images for future review.
Furthermore, digital microscopy adds high-resolution and spatial information that flow measurements cannot extract. Digital slides are also, by nature, more comfortable to share than physical slides thus increasing the possibility of consultations between two or more experts. Digital slides have also the potential to be numerically analyzed directly by computer algorithms, useful to automate the manual counting of structures, or for classifying the condition of the tissue. The extraction of image-based information by computer technology from the digital slide is also known as \textit{Digital Pathology} and can be used both to speed up the process of diagnosis and to reduce uncertainty and subjectivity.

In the last few years, many Computer Aided Diagnosis (\acs{CAD}) system have appeared to automate or aid some stages of the diagnostic process, also motivated by the presence of equipment which allows obtaining slides with good quality automatically. However, automatic interpretation of microscopy medical images is still an open research question. In particular, the primary challenge when developing CAD systems is the creation of an effective method to extract meaningful information from the images, such as the cells number in the film or the position of the different structures in a tissue. These issues become more complex, concerning artificial vision, considering that there is not a color standardization for the staining and acquisition of digital slides. There is a considerable color variability between different slides, due to the quality of the biological sample and the sample preparation, such as the quantity of dye used during the staining procedure, or due to different acquisition system and the image capturing parameters, such as the environment illumination. Furthermore, such variability may be present in the same slide, in particular, the presence of uneven lighting, with a central area very bright and shading areas more marked towards the corners.  Excessive use of the microscope light can cause this problem.

\section{Contributions} 
In this thesis, there is an in-depth analysis of the unsolved issues in Computer-Aided Diagnosis from digital microscopy images, mainly acquired from peripheral blood smears. Different solutions have been analyzed and proposed. Three study cases can be distinguished: White Blood Cells analysis with leukemia correlation, Red Blood Cells analysis with malaria parasites correlation and histological tissues analysis. Particular attention has been given to the extraction of useful information from the digital images and the development of dataset-independent algorithms. In particular, the proposed framework has been tested over well known public datasets for what concerns WBC analysis, while a comprehensive dataset for RBC cells analysis has been introduced and published by our own. Cells count and clumps separation have been addressed in our studies. A crucial step in this kind of work certainly regards the segmentation step, which has several issues: some algorithms, able to isolate the cells of interest from images acquired in different illumination condition and stained with distinct staining have been realized. A correct segmentation step and a subsequent count of the cells permit to manage each cell singularly, and then to diagnose the presence of leukemia, in WBC analysis, or malaria, in RBC analysis. Since the importance of this kind of diagnosis different ensembles of descriptors and classifiers have been evaluated to provide a result as accurate as possible. In the proposed framework no object detection or segmentation method is needed, since every segmentation algorithm, applied to histological images, can produce a considerable number of regions and structures, which is extremely difficult to manage singularly. The overall procedure instead is based on the textures analysis, being the most suitable to analyze the tissue structure. Also, great importance has been given to the analysis of colors, considered one of the most interesting contents to be examined in the histological images, studying not only the internal correlation of various colors but also by analyzing the relationship between different colors.
Moreover, further work has been realized to extend the WBC analysis framework and to include an in-depth RBC analysis, in particular, devoted to malaria parasites analysis. It aims to propose the first public dataset of blood samples afflicted by malaria, specifically designed to evaluate and compare algorithms for segmentation and classification of malaria parasite species. Every image is provided with its related ground truth and parasite's classification of type and stage of life. The primary purpose is to offer a new comparative test tool to the image processing and pattern matching communities, to encourage and improve computer-aided malaria parasites analysis.
The scientific results obtained during this Ph.D. work and described in this thesis also appeared in related publications, following listed:
\begin{itemize}
	\item C. Di Ruberto, A. Loddo, L. Putzu, "A Multiple Classifier Learning by Sampling System for White Blood Cells Segmentation", G. Azzopardi, N. Petkov Eds. Computer Analysis of Images and Patterns - 16th International Conference, CAIP 2015, Valletta, Malta, September 2-4, 2015 Proceedings, Part I. Lecture Notes in Computer Science 9256, Springer 2015, pp. 415-425, ISBN 978-3-319-23191-4.
	\item C. Di Ruberto, A. Loddo, L. Putzu, " Learning by Sampling for White Blood Cells Segmentation", V. Murino, E. Puppo Eds.: Image Analysis and Processing - ICIAP 2015 - 18th International Conference, Genoa, Italy, September 7-11, 2015, Proceedings, Part I. Lecture Notes in Computer Science 9279, Springer 2015, pp. 557-567,  ISBN 978-3-319-23230-0. 
	\item C. Di Ruberto, A. Loddo, L. Putzu. Peripheral Blood Image Analysis. Proceedings of the Doctoral Consortium, 11th Joint Conference on Computer Vision, Imaging and Computer Graphics Theory and Applications, VISIGRAPP 2016; Pages 15-23.
	\item C. Di Ruberto, A. Loddo, L. Putzu. A Leucocytes Count System from Blood Smear Images: Segmentation and Counting of White Blood Cells based on Learning by Sampling. Machine Vision and Applications; Volume 27, Issue 8, November 2016, Pages 1151-1160.
	\item C. Di Ruberto, A. Loddo, L. Putzu., G. Fenu A Computer-Aided System for Differential Count from Peripheral Blood Cell Images. Proceedings of the 12th International Conference on Signal Image Technology \& Internet-Based Systems; 2016, Pages 112-118.
	\item A. Loddo, C. Di Ruberto, L. Putzu Histological Image Analysis by Invariant Descriptors. Proceedings of the 19th International Conference on Image Analysis and Processing, ICIAP 2017; LNCS 2017, vol. 10484, Pages 345-356.
	\item S. Porcu, C. Di Ruberto, A. Loddo, L. Putzu. White Blood Cells Counting Via Vector Field Convolution Nuclei Segmentation. Proceedings of the 13th International Joint Conference on Computer Vision, Imaging and Computer Graphics Theory and Applications, Vol. 4: VISAPP, 227-234, 2018, January 27-29, 2018 Funchal, Madeira, Portugal.
	\item A. Loddo, C. Di Ruberto, M. Kocher. Recent Advances of Malaria Parasites Detection Systems Based on Mathematical Morphology. Sensors. Volume 18, number 2, Pages 513, 2018.
	\item A. Loddo, C. Di Ruberto, M. Kocher, Guy Prod'Hom. MP-IDB: The Malaria Parasite image database for image processing and analysis. SaMBa workshop, MICCAI 2018.
\end{itemize}

\section{Dissertation structure} 
This dissertation describes the work mentioned above in detail. It is organized as follows:
\begin{itemize}
	\item Part \ref{uno} illustrates the phases of a typical CAD system schema for digital microscope images analysis. Four main steps have been highlighted and then analyzed in detail in chapters 2, 3, 4 and 5, namely pre-processing, segmentation, feature extraction, and classification respectively. We give an idea of the most used techniques and illustrating the basic concepts useful to the comprehension of the proposed CAD systems.
	
	\item Part \ref{due} describes the proposed CAD system for peripheral blood image analysis. Chapter 6 contains an extended background of the topics and materials this dissertation takes on, that is a background on hematology and peripheral blood cells, a description of ALL and malaria analysis with references to previous works and to the datasets used to experiment possible solutions. Then chapter 7 shows three different approaches for segmentation, chapter 8 illustrates a method for leukocyte identification and count, while chapter 9 explains the system extension to erythrocyte segmentation to realize a complete blood cells segmenter.
	
	\item Part \ref{tre} concludes the dissertation and gives some final comments on the proposed approaches, discussing the choices made with the obtained results. The experimental results obtained have brought to further ideas for the future, not just to improve the overall procedures but also to extend the proposed CAD systems to further issues and to different medical problems. 
\end{itemize}
Appendix \ref{appendix} outlines the types of diseases that are directly connected with an abnormal number of cells in the peripheral bloodstream and shows how the presence of diseases or parasites may affect the morphology of the cells themselves.