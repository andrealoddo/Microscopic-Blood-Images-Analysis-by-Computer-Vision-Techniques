\chapter*{Abstract}	
\label{chap:global_abstract}

Automatic analysis and information extraction from an image is still a highly challenging research problem in the computer vision area, attempting to describe the image content with computational and mathematical techniques. Moreover, the information extracted from the image should be meaningful and as most discriminatory as possible, since it will be used to categorize its content according to the analyzed problem. In the Medical Imaging domain, many important decisions that affect patient care depends on the usefulness of the information extracted from the image. Managing medical image is even more complicated not only due to the importance of the problem but also because it needs a fair amount of prior medical knowledge to be able to represent with data the visual information to which pathologist refer. 

Today medical decisions that impact patient care rely on the results of laboratory tests to a greater extent than ever before, due to the marked expansion in the number and complexity of offered tests. These developments promise to improve the care of patients, but the more increase the number and complexity of the tests, the more improvements the possibility to misapply and misinterpret the test themselves, leading to inappropriate diagnosis and therapies.  Moreover, pathologists devote much time to the analysis of the tests rather than to the patients care and to the prescription of the right treatment, because of the increased number of tests and amount of data to analyze. Sometimes it can be a waste of time, considering that most of the tests performed are only check-up tests, and most of the analyzed samples come from healthy patients.

Then, a quantitative evaluation of medical images is essential to overcome uncertainty and subjectivity, but also to reduce the amount of data and the timing of the analysis significantly. In the last few years, many computer-assisted diagnosis systems have been developed, attempting to mimic pathologists by extracting features from the images. Image analysis involves complex algorithms to identify and characterize cells or tissues using image pattern recognition technology. This thesis addresses the main problems associated to the digital microscopy analysis in hematology diagnosis, with the development of algorithms both for the extraction of useful information from different digital images and for the distinction of various biological structures. The proposed methods aim to improve the degree of accuracy of the analysis and to reduce the diagnosis time. Furthermore, they can be used as standard tools for skimming the number of samples to be analyzed directly from the pathologist, or as double check systems to verify the correct results of the automated facilities used today.